\documentclass[a4paper,11pt]{scrreprt}
\usepackage{scrhack} % eliminates warning generated by combining scrreprt and listings
\usepackage[utf8]{inputenc}
\usepackage[T1]{fontenc}
\usepackage{listings}
\usepackage[noadjust]{marginnote}
\usepackage{lmodern}
\usepackage{beramono}
\usepackage{fixltx2e,microtype,ellipsis,ragged2e}
\usepackage{graphicx}
\usepackage{pifont}
\usepackage[scale=0.8]{geometry}
\usepackage{color}
\usepackage{MnSymbol}
\usepackage{relsize}

\reversemarginpar

\definecolor{commentColor}{rgb}{0,0.6,0}
\definecolor{stringColor}{rgb}{0.58,0,0.82}

\lstset{%
    basicstyle=\ttfamily\footnotesize,%
    breaklines=true,%
    breakatwhitespace=true,%
    postbreak=\raisebox{0ex}[0ex][0ex]{\ensuremath{\rcurvearrowse\space}},%
    commentstyle=\color{commentColor},%
    keywordstyle=\bfseries\color{blue},%
    stringstyle=\color{stringColor}%
}
\lstloadlanguages{[11]C++,[5.2]Lua}

\def\ifmonospace{\ifdim\fontdimen3\font=0pt}
\def\Cpp{%
\ifmonospace%
    C++%
\else%
    \mbox{C\kern-.1667em\raise.30ex\hbox{\smaller{++}}}%
\fi%
\xspace}

\lstnewenvironment{cpp}
{\lstset{language={[11]C++},title=\Cpp}}
{}

\lstnewenvironment{cppBox}[1][]
{\noindent\minipage[t]{0.48\linewidth}%
\lstset{language={[11]C++},title=\Cpp,#1}}
{\endminipage}

\lstnewenvironment{lua}
{\lstset{language={[5.2]Lua},title=Lua}}
{}

\lstnewenvironment{luaBox}[1][]
{\noindent\minipage[t]{0.48\linewidth}%
\lstset{language={[5.2]Lua},title=Lua,#1}}
{\endminipage}

\newcommand{\marginMarker}[1]{%
\marginnote{%
    \hfill%
		\Huge{#1}%
}[3\baselineskip]%
}

\newcommand{\conforming}{%
\marginMarker{\textcolor{green}{\ding{51}}}%
}

\newcommand{\nonconforming}{%
\marginMarker{\textcolor{red}{\ding{55}}}%
}

\begin{document}
\title{Illarion Coding Style\\0.0}
\author{vilarion \thanks{ \texttt{vilarion@illarion.org, http://www.illarion.org}}}
\date{2014}
\maketitle
\tableofcontents

\chapter{Introduction}
The team of Illarion developers has grown over time. And while everyone has their preferred coding style, it is useful to agree on a single one. This makes it easier for others to read your code, understand and modify it. These rules do not only cover the appearance of code, but also how the coding itself is performed, which functionality to use or avoid and best practices. The purpose of this document is not to hamper you in your coding efforts with an overly exhaustive set of rules, but to guide you towards clean code for the benefit of all.

There are many languages involved in Illarion development. They range from mark-up languages like HTML or the typesetting language \LaTeX\ to a whole array of different programming languages like \Cpp, Java, Lua, PHP or GLSL. Examples given, either conforming or non-conforming, will be in \Cpp and Lua. You can easily derive how they would look like in the language you are currently using. Rules are designed to be as general as possible to apply to most languages used in Illarion development.

Examples are tagged with icons to annotate whether or not they are conforming. Non-conforming examples are designed such, that only the current issue is being violated. They are correct in every other aspect, so that you can focus on the issue at hand.

This is what correct examples look like:

\conforming{}
\begin{cppBox}
for (auto &item: items) {
    item.age();
}
\end{cppBox}
\begin{luaBox}
for _, item in pairs(items) do
    item.age()
end
\end{luaBox}

A violating example looks like this:

\nonconforming{}
\begin{cppBox}
for (auto &item: items)
{
    item.age();
}
\end{cppBox}
\begin{luaBox}
for _, item in pairs(items)
do
    item.age()
end
\end{luaBox}

\chapter{Appearance}
\section{Write lines not longer than 120 characters}
Long lines of code tend to get difficult to read. If your lines get longer than 120 characters, reconsider what you are writing. In fact, most statements should fit into a line of 80 characters. Maybe there is a better way to express what you want to do. If you still need longer lines, break them up, indenting them further than the next statement.

This rule does not apply to continuous text e.g. in HTML. Breaking up continuous text can lead to strange diffs where changing a single word could put a whole paragraph into the diff.

\section{Write one statement per line}
On the other hand, do not cram statements into one line until it reaches its length limit. One statement per line allows for good readability. Compare:

\nonconforming{}
\begin{cppBox}
auto strength = 10; auto wisdom = 8;
\end{cppBox}
\begin{luaBox}
local strength = 10 local wisdom = 8
\end{luaBox}

\conforming{}
\begin{cppBox}
auto strength = 10;
auto wisdom = 8;
\end{cppBox}
\begin{luaBox}
local strength = 10
local wisdom = 8
\end{luaBox}

\section{Frame code blocks with one line on each end}
Each local code block should be framed by one line on each end:

\conforming{}
\begin{cppBox}
for (auto &item: items) {
    item.age();
}
\end{cppBox}
\begin{luaBox}
for _, item in pairs(items) do
    item.age()
end
\end{luaBox}

\nonconforming{}
\begin{cppBox}
for (auto &item: items)
{
    item.age();
}
\end{cppBox}
\begin{luaBox}
for _, item in pairs(items)
do
    item.age()
end
\end{luaBox}

\section{Seperate adjacent code blocks with an empty line}
You can also use an empty line to structure logical groups.

\conforming{}
\begin{cppBox}
auto minutes = 3;
auto seconds = minutes * 60;

for (auto &item: items) {
    for (auto i = 0; i < 5; ++i) {
        item.age(seconds);
    }
}
\end{cppBox}
\begin{luaBox}
local minutes = 3
local seconds = minutes * 60

for _, item in pairs(items) do
    for i = 1, 5 do
        item.age(seconds)
    end
end
\end{luaBox}

\nonconforming{}
\begin{cppBox}
auto minutes = 3;
auto seconds = minutes * 60;
for (auto &item: items) {

    for (auto i = 0; i < 5; ++i) {
        item.age(seconds);
    }
}
\end{cppBox}
\begin{luaBox}
local minutes = 3

local seconds = minutes * 60
for _, item in pairs(items) do
    for i = 1, 5 do
        item.age(seconds)
    end
end
\end{luaBox}

\section{Space}
\subsection{Indent four spaces inside code blocks}
Local code blocks have to be indented by four spaces. Do not use tabulators.

\conforming{}
\begin{cppBox}[showspaces]
{
    auto result = database.query();
    result.validate();
}
\end{cppBox}
\begin{luaBox}[showspaces]
do
    local npc = getNPC()
    npc:talk("Hello!")
end
\end{luaBox}

\subsection{Use space after a delimiter}
Each delimiting character has to be followed by exactly one space:

\conforming{}
\begin{cppBox}[showspaces]
for (const auto number: {42, 0, 1}) {
    cout << number;
}
\end{cppBox}
\begin{luaBox}[showspaces]
for number in {1, 2, 3} do
    print(number)
end
\end{luaBox}

\nonconforming{}
\begin{cppBox}[showspaces]
for (const auto number:{42, 0, 1}) {
    cout << number;
}
\end{cppBox}
\begin{luaBox}[showspaces]
for number in {1,2,3} do
    print(number)
end
\end{luaBox}

\subsection{Use space before and after binary operators}
All binary operators have to be pre- and succeeded by one space:

\conforming{}
\begin{cppBox}[showspaces]
cout << 1;
auto i = 0;
i += 2;
auto b = 3 * i;
\end{cppBox}
\begin{luaBox}[showspaces]
print("a" .. "b")
local i = 0
i = i + 2
local b = 3 * i
\end{luaBox}

\nonconforming{}
\begin{cppBox}[showspaces]
cout<<1;
auto i=0;
i +=2;
auto b =3*  i;
\end{cppBox}
\begin{luaBox}[showspaces]
print("a".."b")
local i =0
i =i+ 2
local b= 3  * i
\end{luaBox}

\subsection{Avoid space after function names}
This speaks for itself:

\conforming{}
\begin{cppBox}[showspaces]
std::move(object);
\end{cppBox}
\begin{luaBox}[showspaces]
print(1)
\end{luaBox}

\nonconforming{}
\begin{cppBox}[showspaces]
std::move (object);
\end{cppBox}
\begin{luaBox}[showspaces]
print (1)
\end{luaBox}

\subsection{Avoid trailing white space}
White space at the end of a line will produce strange diffs, so make sure you do not insert any:

\section{Use camelCase for variable and function names}
Use camelCase for variable and function names:

\conforming{}
\begin{cppBox}
int playerCounter;
Player destroyerOfWorlds;
void fooBar();
\end{cppBox}
\begin{luaBox}
local playerCounter
local robertOppenheimer

function fooBar()
end
\end{luaBox}

\nonconforming{}
\begin{cppBox}
int playercounter;
Player destroyer_Of_Worlds;
void foo_bar();
\end{cppBox}
\begin{luaBox}
local playercounter
local robert_Oppenheimer

function foo_bar()
end
\end{luaBox}


\section{Use TitleCase for class names}
\section{Use proper British English}
\section{Use block comments for copyright header}


\chapter{Coding}
\section{Select expressive names}
\section{Write only useful comments}
\section{Avoid deprecated code}
\section{Avoid duplicating code}
\section{Use libraries}
\section{Keep it as local as possible}
\section{Write short, expressive functions}
\section{Avoid unnecessary code}
\section{Refrain from using goto}
\section{Use correct copyright headers}

\end{document}
